\documentclass[11pt]{article}
\usepackage{amsmath,amssymb,amsthm}
\usepackage{mathtools}
\usepackage{tikz}
\usepackage{enumitem}
\usepackage[margin=1in]{geometry}

\newtheorem{theorem}{Theorem}
\newtheorem*{theorem*}{Theorem}
\newtheorem{lemma}{Lemma}
\newtheorem{definition}{Definition}
\newtheorem*{definition*}{Definition}
\newtheorem{proposition}{Proposition}
\newtheorem{assumption}{Assumption}
\newtheorem{remark}{Remark}



\title{Math 106 -- Notes \\ Week 4: January 26, 2026}
\author{Ethan Levien}
\date{}

\begin{document}
\maketitle


\section*{General continuous time Markov processes (5.3)}

Now we switch back to talking about Markov processes. Eventually we will answer the question: Which Gaussian processes are also Markov processes. A Markov process is defined as follows. 
\begin{definition}[Markov processes]
Let $(\Omega,{\mathcal F},{\mathbb P})$ be a probability space. A process $X_t$ is a Markov processes  if it is ${\mathcal F}_t$-adopted processes and for $s \le t$ and $B \in {\mathcal R}$
\begin{equation}
 {\mathbb P}(X_t \in B|{\mathcal F_s}) =  {\mathbb P}(X_t \in B|X_s)
\end{equation}
\end{definition}

We will consider homogenous Markov processes which are defined by a transition function
\begin{equation}
p(t,x,B) =  {\mathbb P}(X_t \in B|X_0 = x)
\end{equation}
we also define an linear operator 
\begin{equation}
T_tf(x) = {\mathbb E}[f(X_t)|X_0 = x] 
\end{equation}
If there is a density $\rho(t,x,y)$ such that $p(t,x,[y,y+dy))  \approx \rho(t,x,y)dy$, then this becomes
\begin{equation}
T_tf(x) = \int_B f(y) \rho(t,x,y)dy
\end{equation}
A considerable amount of time will later be spent studying the PDE for $\rho$, but first we spend a bit more time on the general case. Before proceeding, it is helpful to note that $p(t,x,B)$ and $T_tf(x)$ correspond respectively to the notation $P_{i,j}(t)$ and $h_{i}(t)$ which was introduced for $Q$-processes. 





\end{document}