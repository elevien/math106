\documentclass[11pt]{article}
\usepackage{amsmath,amssymb,amsthm}
\usepackage{mathtools}
\usepackage{tikz}
\usepackage{enumitem}
\usepackage[margin=1in]{geometry}

\newtheorem{theorem}{Theorem}
\newtheorem*{theorem*}{Theorem}
\newtheorem{lemma}{Lemma}
\newtheorem{definition}{Definition}
\newtheorem*{definition*}{Definition}
\newtheorem{proposition}{Proposition}
\newtheorem{remark}{Remark}



\title{Math 106 -- Notes \\ Week 1: January 9, 2026}
\author{Ethan Levien}
\date{}

\begin{document}
\maketitle

\section*{Time Reversal and Detailed Balance (3.8)}

Let $\{X_t\}_{t \ge 0}$ be a $Q$-process on $\Omega = \{1,\dots,I\}$ initialized in a steady state $\pi$, and define the time-reversed process
\begin{equation}
\hat{X}_t = X_{T-t},
\end{equation}
where $T>0$ is arbitrary (since $X$ is initialized in a steady state). We seek to describe the process $\hat{X}_t$. Using Bayes' theorem, we can easily see that it is a Markov process whose transition matrix $\hat{P}_{i,j}(t)$ is given by 
\begin{equation}
\hat{P}_{i,j}(t) 
= \mathbb{P}(X_{s-t}=j \mid X_s = i) 
= \mathbb{P}(X_s=i \mid X_{s-t} = j)\,\frac{\mathbb{P}(X_{s-t} = j)}{\mathbb{P}(X_s = i)}
= P_{j,i}(t)\,\frac{\pi_j}{\pi_i}.
\end{equation}

\subsection*{Generator of the time-reversed process}

Let $\hat{Q}$ denote the generator of $\hat{X}$. By the {\bf backward equation}
\begin{equation}
\frac{d}{dt}\hat{P}_{i,j}(t) = \sum_{k=1}^I\hat{Q}_{i,k} \hat{P}_{k,j}(t)
\end{equation}
On the other hand, using the {\bf forward equation} for $X$, 
\begin{align}
\frac{d}{dt}\hat{P}_{i,j}(t) 
&= \frac{d}{dt}\Big(P_{j,i}(t)\frac{\pi_j}{\pi_i}\Big)
= \sum_{k=1}^I P_{j,k}(t)Q_{k,i}\,\frac{\pi_j}{\pi_i} 
= \sum_{k=1}^I \left( P_{j,k}(t)\frac{\pi_j}{\pi_k}\right) \left( Q_{k,i} \frac{\pi_k}{\pi_i} \right)\\
&= \sum_{k=1}^I \hat{Q}_{i,k}\hat{P}_{k,j}(t)
\quad \text{ with }\quad 
\hat{Q}_{i,j} = \frac{\pi_j}{\pi_i}Q_{j,i}.
\end{align}
\subsection*{Detailed balance and probability fluxes}

The chain is said to satisfy \emph{detailed balance} with respect to $\pi$ if
\begin{equation}
\pi_i Q_{i,j} = \pi_j Q_{j,i}, \qquad \text{for all } i\neq j.
\end{equation}
Comparing with the formula for $\hat{Q}$, we see that detailed balance is equivalent to
\begin{equation}
\hat{Q}_{i,j} = Q_{i,j} \quad \text{for all } i,j.
\end{equation}
Thus, a stationary $Q$-process is reversible if and only if it satisfies detailed balance; in that case the time-reversed process has the same generator as the forward process.

To interpret detailed balance, note that in stationarity the average rate at which the process makes transitions from state $i$ to state $j$ is
\begin{equation}
J_{i\to j} \coloneqq \pi_i Q_{i,j}.
\end{equation}
This quantity has the interpretation of a \emph{probability flux} along the directed edge $i\to j$. The net flux along the undirected edge $\{i,j\}$ is
\begin{equation}
J_{i\leftrightarrow j}^{\mathrm{net}}
= J_{i\to j} - J_{j\to i}
= \pi_i Q_{i,j} - \pi_j Q_{j,i}.
\end{equation}
The detailed balance condition
is precisely the requirement that 
\begin{equation}
J_{i\leftrightarrow j}^{\mathrm{net}} = 0 \quad \text{for every pair } i,j,
\end{equation}
i.e.\ the flux from $i$ to $j$ is exactly balanced by the flux from $j$ to $i$ on each edge. 

Mere stationarity of $\pi$ only requires
\begin{equation}
\sum_{j=1}^I \pi_i Q_{i,j} = 0 \quad \text{for each } i,
\end{equation}
which says that the total outgoing flux from each state $i$ is balanced by the total incoming flux, but allows for nonzero \emph{circulating currents} around cycles. Detailed balance rules out such circulating probability currents; at equilibrium the chain looks statistically the same when run forwards or backwards in time.

\subsection*{Hermitian structure}

Assume that $Q$ is reversible with respect to $\pi$, i.e.\ it satisfies detailed balance. Let $D_\pi$ be the diagonal matrix
\begin{equation}
(D_\pi)_{i,i} = \pi_i, \qquad (D_\pi)_{i,j}=0 \text{ for } i\neq j.
\end{equation}
Consider the similarity transform
\begin{equation}
S \coloneqq D_\pi^{1/2}\,Q\,D_\pi^{-1/2}.
\end{equation}
The $(i,j)$-entry of $S$ is
\begin{equation}
S_{i,j}
= (D_\pi^{1/2})_{i,i} Q_{i,j} (D_\pi^{-1/2})_{j,j}
= \sqrt{\pi_i}\,Q_{i,j}\,\frac{1}{\sqrt{\pi_j}}.
\end{equation}
Using detailed balance,
\begin{equation}
\pi_i Q_{i,j} = \pi_j Q_{j,i}
\quad \Longrightarrow \quad
\sqrt{\pi_i}\,Q_{i,j}\,\frac{1}{\sqrt{\pi_j}}
= \sqrt{\pi_j}\,Q_{j,i}\,\frac{1}{\sqrt{\pi_i}},
\end{equation}
so
\begin{equation}
S_{i,j} = S_{j,i}.
\end{equation}
Hence $S$ is real symmetric.

Because $Q$ is similar to the symmetric matrix $S$, it follows that $Q$ is diagonalizable with real eigenvalues. More precisely, there exists an invertible matrix $S$ and a real diagonal matrix $\Lambda$ such that
\begin{equation}
Q = S \Lambda S^{-1}, \qquad \Lambda = \mathrm{diag}(\lambda_1,\dots,\lambda_I), \quad \lambda_k \in \mathbb{R}.
\end{equation}
For a finite, irreducible $Q$-process, $0$ is a simple eigenvalue with eigenvector $\mathbf{1}$, and all other eigenvalues satisfy
\begin{equation}
\lambda_k < 0 \quad \text{for } k\ge 2.
\end{equation}
The transition semigroup then has the spectral representation
\begin{equation}
P(t) = e^{tQ} = S\,e^{t\Lambda}\,S^{-1},
\end{equation}
so each nontrivial mode decays as a pure exponential $e^{\lambda_k t}$ with a real decay rate $\lambda_k<0$. In particular, an equilibrium $Q$-process (one satisfying detailed balance) does not relax via oscillations: there are no complex-conjugate pairs of eigenvalues, and thus no oscillatory components in the relaxation dynamics; all deviations from equilibrium decay monotonically in time along real exponential modes.
\end{document}

\section*{Appendix: Basic facts about real symmetric matrices (optional)}

We collect here a few standard facts about real symmetric matrices used above, together with proofs.

\subsection*{Real eigenvalues}

\begin{lemma}
If $A \in \mathbb{R}^{I\times I}$ is symmetric, $A = A^\top$, then every eigenvalue of $A$ is real.
\end{lemma}

\begin{proof}
Let $\lambda \in \mathbb{C}$ and $v \in \mathbb{C}^I\setminus\{0\}$ satisfy
\begin{equation}
A v = \lambda v.
\end{equation}
Consider the standard Hermitian inner product $\langle x,y\rangle = x^* y$ on $\mathbb{C}^I$ (where $x^*$ denotes conjugate transpose). Then
\begin{equation}
\langle Av, v\rangle = \langle \lambda v, v\rangle = \lambda \langle v, v\rangle.
\end{equation}
On the other hand, since $A$ is real symmetric, $A = A^\top = A^*$, so
\begin{equation}
\langle Av, v\rangle = \langle v, A^* v\rangle = \langle v, Av\rangle = \langle v, \lambda v\rangle = \overline{\lambda}\,\langle v, v\rangle.
\end{equation}
Thus
\begin{equation}
\lambda \langle v, v\rangle = \overline{\lambda}\,\langle v, v\rangle.
\end{equation}
Since $v\neq 0$, we have $\langle v, v\rangle > 0$, hence $\lambda = \overline{\lambda}$, i.e.\ $\lambda$ is real.
\end{proof}

\subsection*{Orthogonal diagonalization (spectral theorem)}

\begin{theorem}[Spectral theorem]
If $A \in \mathbb{R}^{I\times I}$ is symmetric, then there exists an orthogonal matrix $U$ ($U^\top U = I$) and a real diagonal matrix $\Lambda$ such that
\begin{equation}
A = U \Lambda U^\top.
\end{equation}
\end{theorem}

\begin{proof}
We sketch a standard proof by induction on $I$.

For $I=1$ the statement is trivial. Assume the theorem holds for $(I-1)\times(I-1)$ real symmetric matrices. Let $A \in \mathbb{R}^{I\times I}$ be symmetric. By the previous lemma, $A$ has a real eigenvalue $\lambda_1$ with real eigenvector $v_1\neq 0$. Normalize $v_1$ so that $\|v_1\|=1$. Extend $\{v_1\}$ to an orthonormal basis $\{v_1,\dots,v_I\}$ of $\mathbb{R}^I$ (e.g.\ by Gram–Schmidt). Let $U_1$ be the orthogonal matrix whose columns are $v_1,\dots,v_I$.

In the basis given by $U_1$, the matrix of $A$ is
\begin{equation}
B \coloneqq U_1^\top A U_1.
\end{equation}
By construction, $B$ has the block form
\begin{equation}
B = \begin{pmatrix}
\lambda_1 & 0 \\
0 & A'
\end{pmatrix},
\end{equation}
where $A'$ is an $(I-1)\times(I-1)$ real symmetric matrix (symmetry is preserved under orthogonal similarity). By the induction hypothesis, there exists an orthogonal $(I-1)\times(I-1)$ matrix $U'$ and a real diagonal $(I-1)\times(I-1)$ matrix $\Lambda'$ such that
\begin{equation}
A' = U' \Lambda' U'^\top.
\end{equation}
Define
\begin{equation}
U_2 \coloneqq 
\begin{pmatrix}
1 & 0 \\
0 & U'
\end{pmatrix},
\qquad
\Lambda \coloneqq 
\begin{pmatrix}
\lambda_1 & 0 \\
0 & \Lambda'
\end{pmatrix}.
\end{equation}
Then $U_2$ is orthogonal and
\begin{equation}
U_2^\top B U_2 
= 
\begin{pmatrix}
1 & 0 \\
0 & U'^\top
\end{pmatrix}
\begin{pmatrix}
\lambda_1 & 0 \\
0 & A'
\end{pmatrix}
\begin{pmatrix}
1 & 0 \\
0 & U'
\end{pmatrix}
=
\begin{pmatrix}
\lambda_1 & 0 \\
0 & \Lambda'
\end{pmatrix}
= \Lambda.
\end{equation}
Setting $U \coloneqq U_1 U_2$, which is orthogonal as a product of orthogonal matrices, we obtain
\begin{equation}
A = U \Lambda U^\top.
\end{equation}
This completes the induction.
\end{proof}

\subsection*{Orthogonality of eigenvectors}

\begin{lemma}
Let $A \in \mathbb{R}^{I\times I}$ be symmetric. If $v,w \in \mathbb{R}^I$ are eigenvectors with distinct eigenvalues $\lambda,\mu \in \mathbb{R}$, i.e.
\begin{equation}
A v = \lambda v, \qquad A w = \mu w, \qquad \lambda \neq \mu,
\end{equation}
then $v$ and $w$ are orthogonal:
\begin{equation}
v^\top w = 0.
\end{equation}
\end{lemma}

\begin{proof}
Compute
\begin{equation}
\lambda\, v^\top w = (A v)^\top w = v^\top A^\top w = v^\top A w = v^\top (\mu w) = \mu\, v^\top w.
\end{equation}
Thus
\begin{equation}
(\lambda - \mu) v^\top w = 0.
\end{equation}
Since $\lambda \neq \mu$, it follows that $v^\top w = 0$.
\end{proof}

In particular, eigenvectors corresponding to distinct eigenvalues can be chosen to form an orthonormal basis of $\mathbb{R}^I$, as in the spectral theorem.

\subsection*{A.4 Matrix functions and the exponential}

Let $A \in \mathbb{R}^{I\times I}$ be symmetric. By the spectral theorem, there exists an orthogonal matrix $U$ and a diagonal matrix $\Lambda = \mathrm{diag}(\lambda_1,\dots,\lambda_I)$ such that
\begin{equation}
A = U \Lambda U^\top.
\end{equation}
For any function $f:\mathbb{R}\to\mathbb{R}$ for which $f(\lambda_k)$ is defined for all eigenvalues $\lambda_k$, we define
\begin{equation}
f(A) \coloneqq U\,f(\Lambda)\,U^\top,
\end{equation}
where $f(\Lambda)$ is the diagonal matrix
\begin{equation}
f(\Lambda) = \mathrm{diag}(f(\lambda_1),\dots,f(\lambda_I)).
\end{equation}
One checks directly that this definition is independent of the particular orthogonal diagonalization and that $f$ respects algebraic relations satisfied by $A$.

In particular, for the matrix exponential, defined by the convergent power series
\begin{equation}
e^{tA} \coloneqq \sum_{n=0}^\infty \frac{t^n}{n!} A^n,
\end{equation}
we obtain
\begin{equation}
e^{tA} = U\,e^{t\Lambda}\,U^\top,
\end{equation}
where
\begin{equation}
e^{t\Lambda} = \mathrm{diag}(e^{\lambda_1 t},\dots,e^{\lambda_I t}).
\end{equation}
Thus the dynamics generated by $A$ are a superposition of exponential modes $e^{\lambda_k t}$ along the orthonormal eigenvectors of $A$, as used in the discussion of $e^{tQ}$ above.


\end{document}